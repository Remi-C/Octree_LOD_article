%%
%% Copyright 2011 Clement Mallet IGN Societe Francaise de 
%% Photogrammetrie et de Teledetection
%% ======================================================================
%% WARNING! ENGLISH SPEAKING AUTHORS SHOULD READ rfpt_template_eng.tex
%%          FILE INSTEAD
%% ======================================================================

\documentclass{rfpt}

%% --------------------------------------------------------------
%% FIGURES
%% Pour des figures en PostScript dans votre article
%% utilisez le paquet graphics pour de simples commandes
%% \usepackage{graphics}
%% ou le paquet graphicx pour des commandes plus compliqu�es
%% \usepackage{graphicx}

%% --------------------------------------------------------------
%% BIBLIOGRAPHIE
%% Merci de ne pas utiliser le package cleveref
%% Le package natbib est utilise par defaut, et presente de nombreuses
%% fonctionnalites. Vous pouvez par exemple vous referer ici
%% http://merkel.zoneo.net/Latex/natbib.php?lang=fr

%% --------------------------------------------------------------
%% UTILISATION DE CARACTERES ACCENTUES AU CLAVIER ?
%% (le codage du clavier depend du systeme d'exploitation)
% \usepackage[applemac]{inputenc} % MacOS
% \usepackage[ansinew]{inputenc}  % Windows ANSI
% \usepackage[cp437]{inputenc}    % DOS, page de code 437
% \usepackage[cp850]{inputenc}    % DOS, page de code 850
% \usepackage[cp852]{inputenc}    % DOS, page de code 852
% \usepackage[cp865]{inputenc}    % DOS, page de code 865
%\usepackage[latin1]{inputenc}   % UNIX, codage ISO 8859-1
% \usepackage[decmulti]{inputenc} % VMS
% \usepackage[next]{inputenc}
% \usepackage[latin2]{inputenc}
% \usepackage[latin3]{inputenc}


\usepackage{amssymb,amsthm}

\begin{document}

\begin{frontmatter}

%% Titre, auteurs, affiliation & adresse

%% Garder cet exemple
%% \title{Titre}
%% \author{Pr�nom Nom}
%% \address{Affiliation + adresse}

\title{Instructions to authors submitting to the Revue Fran�aise de Photogramm�trie et de T�l�d�tection -- \LaTeXe format}

\author[A1]{Michel Bertrand}
\author[A1,A2]{Pierre-Jean Dupont}
\author[A3]{Jacques Pierre}

\address[A1]{My laboratory, my address, 00000 Mycity, England}
\address[A2]{Another lab, another address, 00100 Anothercity, Cedex, France}
  \address[A3]{University of City, the place, 00200 Againanothercity, Spain}

\begin{resume}
Les auteurs publiant � la RFPT et utilisant le
traitement de texte \LaTeXe\ trouveront ci-dessous quelques
indications destin�es � leur faciliter la t�che. Le
fichier \texttt{rfpt\_template\_fr.tex} permet la cr�ation de ce document et de respecter les contraintes de style de la revue. Vous pouvez l'utiliser comme base.
\end{resume}
%%%%%%%%


\begin{motscle}
%% Les mots cl�s ici, s�par�s de la commande \sep
Mot cl� 1 \sep Mot cl� 2 \sep Mot cl� 3
\end{motscle}

\begin{abstract}
RFPT authors using\LaTeXe\ will find
below some information to help them. The file \texttt{rfpt\_template\_fr.tex} allows to create this document and respect the layout of the journal. You may use it as a basis.\end{abstract}
%%%%%%%%


\begin{keyword}
%% keywords here, in the form: keyword \sep keyword
Keyword 1 \sep Keyword \sep Keyword 3
\end{keyword}


\end{frontmatter}


%%%%%%%%%%%%
\section{Document format}
%%%%%%%%%%%%
\subsection{\texttt{rfpt} class}
The submitted papers have no page limitation. We advise you to use the
\texttt{rfpt.cls} \LaTeXe\ class file to perform automatic page
setting:
\begin{verbatim}
    \documentclass{rfpt}
\end{verbatim}
In your file preamble, you have to enter the following informations:
\begin{itemize}
    \item paper title :\\
    \verb!\title{Title of you rpaper}!

    \item Firstname and last name of each author, preceeded by a number linking to his address :\\
    \verb!\author[A1]{Michel Bertrand}!\\
    \verb!\author[A1,A2]{Pierre-Jean Dupont}!

    \item the address of each author, including his e-mail address:\\
    \verb!\address[A1]{MyLab, Insitute of RFPT!\\
     \verb!mon address, 00000 Mycity, Spain}! \\


    \item then, french and english written abstracts :\footnote{French
    written abstract is optional, but highly recommended.}\\
    \verb!\begin{resume}R\'esum\'e fran�ais!\\
    \verb!\end{resume}! \\
    \verb!\begin{abstract}English written!\\
    \verb! abstract\end{abstract}!

    \item finally, your text, and the bibliography: \\
    \verb!\begin{document}! \\
    \verb!Body of your paper !\\
    \verb!\bibliographystyle{rfpt_fr}!\\
    \verb!\bibliography{biblio}! \\
    \verb!\end{document}!

\end{itemize}

\subsection{Section and subsection}
This example file uses \verb!\section! and \verb!\subsection!. For
lower level sectioning commands, you obtain:
\subsubsection{Subsubsection}
By means of \verb!\subsubsection!.
\paragraph{Subsubsubsection}
By means of \verb!\paragraph!.

\section{Tables, figures and mathematics}
Number illustrations and tables consecutively in accordance with their appearance in the text
Ensure that each illustration and each table has a caption. The illustrations and tables should be followed by their caption, as illustrated for Table~\ref{power}.\\
The caption should start with an upper case and finish with a point.
\begin{table}[htb]
    \begin{center}
    \begin{tabular}{||c||*{8}{c|}|}
        \hline\hline
        $n$   & 1 & 2 & 3 &  4 &  5 &  6 &   7 &   8 \\ \hline
        $2^n$ & 2 & 4 & 8 & 16 & 32 & 64 & 128 & 256 \\
        \hline\hline
    \end{tabular}
        \caption{\label{power}2 to the power.}
    \end{center}
\end{table}
\begin{figure}[htb]
    \begin{center}
    \setlength{\unitlength}{0.5cm}
    \begin{picture}(5,5)
        \put(2.5,2.5){\oval(5,5)}
        \put(1,1){\line(1,0){3}}
        \put(4,1){\line(0,1){3}}
        \put(1,4){\line(1,0){3}}
        \put(1,1){\line(0,1){3}}
    \end{picture}
    \end{center}
    \caption{A square in an oval.}
    \label{cercle}
\end{figure}

Including \texttt{Postscript} graphics files is easily performed by
means of \texttt{graphics}, \texttt{graphicx} or \texttt{epsfig}
packages.
Mathematical formulas appearence can be improved by means of
\texttt{amsmath} package from $\mathcal{AMS}$-\LaTeX. They
have to be numbered as formula~\ref{formula}:
\begin{equation}
   \label{formula}
   F(x) = \int_{-\infty}^x f(u)\,du
\end{equation}

\section{Bibliography}
Please ensure that every reference cited in the text is also present in the reference list (and vice versa). Any references cited in the abstract must be given in full. 
\subsection{How to cite ?}
\texttt{natbib} package is used to create the References section. You can refer to the documentation available on the Net. You can download \texttt{natbib.sty} file if the compilation of your paper fails.
Then, you have to include it in the current directory.
The two main options are:
\begin{itemize}
    \item "\ldots as illustrated in \citep{companion}", using  \\
    \verb!\citep{companion}!
        \item and "\citet{lamport94a} writes that\ldots" using\\
    \verb!\citet{lamport94a}!
\end{itemize}

\subsection{Format}
\begin{itemize}
\item Authors: they should be separated in the .bib file using the term "and".
\item Journal: volume, number, and pages are mandatory when available.
\item Conference: the address (city+country) has to be specified. The same style should be chosen for all the papers that have been published in the same conference, whatever the year.
\item Web: as a minimum, the full URL should be given and the date when the reference was last accessed
\end{itemize}

\bibliographystyle{rfpt}
\bibliography{biblio}

\end{document}

%%
%% End of file `rfpt_template.tex'.
