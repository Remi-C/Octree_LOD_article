
%% ---------------------------------------------------------------------
%% Copyright 2014, Thales, IGN, Remi Cura
%% 
%% This file list the used package along description as well as other settings like used defined macro
%%
%%NOTE : 
%% ---------------------------------------------------------------------


%%Document structure :
	
	% 1 or 2 columns, chose one of the following
		%\documentclass[preprint,12pt,authoryear,twocolumn]{elsarticle}
		%\documentclass[preprint,3p,12pt,authoryear]{elsarticle}
		%\documentclass{rfpt}
		\documentclass{isprs}
		\usepackage{subfigure}
		\usepackage{setspace}
		\usepackage{geometry}
		\geometry{a4paper, top=25mm, left=20mm, right=20mm, bottom=25mm, headsep=10mm, footskip=12mm}
		
		%encoding
		\usepackage[utf8]{inputenc}
		
	%encoding
%		\usepackage[utf8]{inputenc}
	\usepackage{nameref}
	
	%math 
		\usepackage{amsmath}
		
		%\usepackage[url=false]{biblatex}
	%bibliogrpahy references style
		%%note : to remove URL from citations, comment line 371 of file elsarticle-harv.bst
		\bibliographystyle{isprs} 
		\usepackage{natbib} % necessary to control spacing
		\setlength{\bibsep}{1.5pt} % rcontrolling the spacing between references
		
	%hyperlink reference for inteligent pdf output 
 		\usepackage[pagebackref=true, pdftex]{hyperref}
		\hypersetup{colorlinks=true}

	%for the images
		\usepackage{graphicx}
		\DeclareGraphicsExtensions{.pdf,.png,.jpg}

%%Useful
	%adding more default colors%
		\usepackage[usenames,dvipsnames]{color}

	%allowing to display simple algorithm
		\usepackage{algorithm2e}
	%this package improve the spacing of letters : reduce the number of lines 
		\usepackage{microtype} 
		
	%allow to use SI unit with command
		\usepackage[squaren, Gray, cdot]{SIunits}

	%this package is buggy but provides way to include/exclude comment (used to generate a "debug" version of article)
		\usepackage{comment}
			%chose one of the two to show/hide comments
				\includecomment{proofreading}
				%\excludecomment{proofreading}
				
				\includecomment{illustrationInline}
				%\excludecomment{inline_illustration}
			
			%allow to modify the style of the commande : BUGGY : mess with subparapgraph command
				%\specialcomment{proofreading}{\begingroup\ttfamily\scriptsize\color{Orange}}{\color{black}\endgroup}
				%\specialcomment{proofreading}{\color{Orange}}{\color{black}}
			
	% add a \todo command to insert a todo and list it at the end of the document
		%% Note : chose between the 2 options to make todo inline or in margin
		%note : Change this package and switch to todonotes, more powerfull
		%\usepackage[superscript]{todo}
		%\usepackage[marginpar]{todo}  

	% another todo packages
		\usepackage[french,colorinlistoftodos]{todonotes}
		
	% defining a new command for emphas ein order to be able to desactivate it if necessary
		%%Note : the first option do somehting, chose the second one to do nothing (i.e. print text wihtout emphase)
		\newcommand{\myemph}[1]{\emph{#1}}
		%\newcommand{\myemph}[1]{#1}
	
	%defining new command for todo :
		% option to allow mutli lines in \todo[caption={Short note}
		% option to get arrow instead of lines : fancyline
		% option to precise author of todo : author=Xavier
		
		%redefining todo to add sub section number: 	
			%\newcommand{\ntodo}[2][]{\todo[#1]{\thesubsubsection{}. #2}}
		%use small line spacing :
			%\newcommand{\smalltodo}[2][]
			%{\todo[caption={#2}, #1]
			%{\begin{spacing}{0.5}#2\end{spacing}}}
		
		\newcommand{\mytodo}[2][]
		{\todo[size=\tiny,caption={#2}, #1, inline]
		{RC:\thesubsubsection{}. #2 }
		}
		
		%when a ref is missing or wrong	
		\newcommand{\todoref}[2][] 
			{\mytodo[color=blue!40, #1]
			{#2 }}
		
		%When a inside reference is missing or wrong
		\newcommand{\todorenv}[2][] 
			{\mytodo[color=yellow!40, #1]
			{#2 }}
		
		%When content need a rewrite
		\newcommand{\todorewrite}[2][] 
			{\mytodo[color=red!40, #1]
			{#2 }}
			
		%general todo
				\newcommand{\todoall}[2][] 
					{\mytodo[color=pink!40, #1]
					{#2 }}
					
		%When noting something
		\newcommand{\todonote}[2][] 
			{\mytodo[color=green!40, #1]
			{#2 }}
					
			%color option : blue : ref , green: note , red : rewrite
			
			
	%defining new command for image insertion
		%\usepackage{svg}
		
		\newcommand{\myimage}[3]
					{
					\begin{figure} [!h]
						\begin{center}
							\includegraphics[width=\linewidth,keepaspectratio]{#1}
							\caption{#2}  
							\label{#3}
							\end{center}
					\end{figure} 
					}
					
		\newcommand{\myimageFullPageHeight}[3]
						{
						\begin{figure*} [!h]
							\begin{center}
								\includegraphics[height=\textheight,keepaspectratio ]{#1}
								\caption{#2}  
								\label{#3} 
							\end{center}
						\end{figure*} 
						}
		\newcommand{\myimageFullPageWidth}[3]
								{
								\begin{figure*} [ht!]
									\begin{center}
										\includegraphics[width=\textwidth,keepaspectratio ]{#1}
										\caption{#2}  
										\label{#3} 
										\end{center}
								\end{figure*} 
								}
		
		\newcommand{\myimagelab}[3][] 
					{\begin{figure} [!h]
						\begin{center}
							\includegraphics[width=1\textwidth]{#1}
							\caption{#2}  
							\label{#3}
						\end{center}
					\end{figure} 
					}
					
		\newcommand{\myimageInline}[2]
					{
					\myimage{#1}{#2}
					}	 
		
	%defining a command to inser percent symbol
		%note : should be done using the siunitx package			
		\newcommand{\mypercent}{~\%}	
							
	%defining new command to change typo of remark area:
		\newcommand{\myremark}[1]{
		\color{black!80}
		#1
		\color{black!100}
		}
		%\todoall{trouver le bon style de typo pour les remarques}
	 
%% SETTINGS
	%show N level of hierarchy in table of content
		\setcounter{tocdepth}{3}%%chosing the numbe rof level in rable of contents. Default = 3
		
		\begin{proofreading}
			%show N level of hierarchy in table of content
				\setcounter{tocdepth}{4}%%chosing the numbe rof level in rable of contents. Default = 3
		%setting the size of margin in proof to help reduce page number
				%\usepackage[margin=1in]{geometry}
		\end{proofreading}
		
	
	
	
