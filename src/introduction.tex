%% ---------------------------------------------------------------------
%% Copyright 2014, Thales, IGN, Rémi Cura
%% 
%% This file contains the introduction of article
%% ---------------------------------------------------------------------


\section{Introduction}
	\subsection{Problem} 
	 
		Point cloud data is becoming more and more common. Following the same trend, the acquisition frequency and precision are also increasing.
		Thus point cloud processing is clocking on the Big Data door.
		
		Yet the usage of point cloud data is also spreading and going out of the traditional user communities. 
		Lidar are now commonly used by non-specialized users. 
		
		
		For many usage, having the raw, complete point cloud is unnecessary, or even damageable.
		Thus we deal with a simpler version of a problem that the G.I.S community has faced for a long time : how to generalize point cloud, with data sets that are several order of magnitude bigger than usual vector data set?
		
		It is after all a problematic very common in data processing. Having a big data set, how to reduce its size while preserving its characteristics.
		It is the essence of compression for instance.
		
		Generalization is also more difficult when mixing data set with varying densities. For instance an aerial Lidar map augmented at certain places by terrestrial scanners, or vehicle-based Lidar acquisition, where the density varies with speed and scene geometry.
		
		Here we deal with a simplified version : given a point cloud, how to efficiently generate Level Of Detail (LOD, cf \href{banner_image}{figure ~\ref{fig:banner_image}} ) of this point cloud while preserving the geometric characteristic, without duplicating data?
		The key to LOD approach is efficiency. Indeed LOD approaches sacrifices of a part of information in exchange of a massive reduction of data size. That's why a solution using LOD must by nature be efficient, or the information loss would be pointless.
		
	%	\paragraph{}
	\subsection{Motivation}
		\begin{itemize}
			\item Point cloud : becoming common
				Point cloud are becoming common because sensor are smaller, cheaper, easier to use. Point cloud from image (using Stereo Vision) are also easy to get with several mature structure from motion solutions.
				Point cloud complements very well images, Lidar point cloud allowing to avoid the ill-posed problem of stereo-vision, and providing key data to virtual reality.
			\item Growing data set and Multi sources 
				At such the size of data set are growing, as well as the number of dataset and their diversity.
			\item a now widely use data type
				The point cloud data are now well established in a number of industries, like construction, architecture, robotics, archaeology, as well as all the traditional GIS fields (mapping, survey, cultural heritage)
			\item Much less focus on informatics/storing 
				The LIDAR research community is very active. The focus of Lidar researchers is much more on Lidar processing and Lidar data analysis, or the sensing device, than on methods to render the data size tractable. 
		\end{itemize}   
		
	\subsection{state of the art}
		\todorewrite{when I have access to Zotero} 
		State of the art should include 
		\begin{itemize}
			\item what people do with point cloud ? (oosterom 2014)
		\end{itemize}
	\subsection{discussion of bibliography} 
		%limits of the already published articles
		
		Point cloud generalization methods are far from the subtlety of vector generalization ( // mettre des references).
		
		More generally, proposed methods usually focus on data compression and computing acceleration.
		
		Another common practice coming from Computer Graphics field is to compute an octree over the point cloud, then for each cell , compute a sub-sampled version of the point cloud.
		This allows to have simple and efficient LOD , at the price of data duplication, and high sensibility to density variation.
		
		Moreover, all the methods are specific and depend on a specific data structure that has to be stored extra to the point cloud data.Steaming from this, the interoperability is non existent (moving from one software to another, one loose the data structure).
		
		It is the classical and seemingly intractable trade-off between computing and storage : When pre-computing a data structure, it need to be stored extra. When doing the computing on the fly, the exact same operations many time.
	
	\subsection{contribution}
		
		This paper re-use and combine existing and well established methods with a focus on  simplicity and efficiency. AS such, all the methods are tested on Billions scale point cloud, and are open source for sake of reproducibility test and improvements.
		We propose a simple method that enable portable, computation-free, geometrical Level Of Detail.
		Our first contribution is to propose to store the LOD information directly into the ordering of points rather than externally, avoiding any data duplication.
		Thus, the more we read points, the more precise of an approximation of the point cloud we get. If we read all the points, we have the original point cloud.
		
		The second contribution is a simple way to order points in order to have a increasingly better geometric approximation of the point cloud when following this order.
		
		The third contribution is to use the ordering construction by-product as a simple and free dimensionality descriptor. We demonstrate the interest of this descriptor in a Random forest classification (training-filtering) experiment.
			
		
	\subsection{plan of the article}
		The rest of this article is organized as follow :
		In the next section \ref{sec:method} we present the methods.  
		In the result section \ref{sec:result} we give results and order of magnitude
		. We discuss it and possible limitations in the \ref{sec:discussion}.  
