%% ---------------------------------------------------------------------
%% Copyright 2014, Thales, IGN, Remi Cura
%% 
%% This file gives a list of stuff to do
%%
%%NOTE : 
%% ---------------------------------------------------------------------

\newpage

\section{TODOLIST}

	\subsection{illustrations} 

	\subsection{liens}
	\subsection{contenu}
	\subsection{correction}
	
	
	
	\subsection{en vrac}
	
	Réorganiser le contenu de l’intro pour que la schéma soit plus claire 
	
	Insister sur les problématiques côté urbanisme (vs entertainment)
	
	Ajouter l’idée qu’on est en fait oblige de modéliser 2 fois : une fois comme cible de l’aglorithme de reconstruction, une autre fois parce que c’est imposé par la façon d’afficher : cela introduit des contraintes spécifiques qu’il faut prendre en compte.
	
	Insérer dans les utilisations les contraintes qui sont apportées.
	
	Renvoie vers l état de l art et résumé en 1 page
	
	Distribution des mots clefs par méthode pour montrer les tendances 
	
	Tokyo ou NewYork pour compléter sur un modèle traditionnellement plus large
	
	Augmenter significativement la partie sur les applications 
	
	
		\subsection{Task list}
		
		
	\begin{itemize}\itemsep0.1pt
	\item générale
		relecture, liens , reference au propre (sans URL)
  \item Texte
  \begin{itemize}\itemsep0.1pt
      \item Intro
	        	\begin{itemize}\itemsep0.1pt
	        		\item Ménage dans l’introduction
	        		\item Complétions introduction
	        		\item Annonce du plan (mode liaison)
	        		\item Limite de l’état de l’art a compléter et préciser pour bien montrer la logique
	        	\end{itemize}
      \item Cours
	        	\begin{itemize}\itemsep0.1pt
		        	\item Procedural
		        	\item grammaire
		        	\item inverse procedural modeling 
       \end{itemize}
      \item etat de l'art
	        	\begin{itemize}\itemsep0.1pt
     					\item par objets
     						\begin{itemize}\itemsep0.1pt
  						    \item Street feature
  						    \item buildings
     						\end{itemize}
  				    \item par methode 
     				    	\begin{itemize}\itemsep0.1pt
 				    	      	\item Procedural
    				   			\item Grammaire
    				   			\item inverse procedural modeling 
  			    \end{itemize}
     	 	 \end{itemize}
      \item conclusion
	       	\begin{itemize}\itemsep0.1pt
	       		\item Lien vers mon sujet
	      		\item Perspectives générales
	       		\item resumé
       	 \end{itemize}
  \end{itemize}
  \item Illustration
  \begin{itemize}\itemsep0.1pt  
  		       		\item Screen terra dynamica avec sans objets
  		       		\item Exemple ODParis normal puis avec objets placés aléatoirement
  		       		\item Illustration Jerome D. pour vue capteur
      		\end{itemize}
	\end{itemize}
 

\newpage

%\listoftodos

\newpage
