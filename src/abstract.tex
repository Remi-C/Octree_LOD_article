%% ---------------------------------------------------------------------
%% Copyright 2014, Thales, IGN, Rémi Cura
%% 
%% This file is the abstract of the article
%% ---------------------------------------------------------------------
 
%\begin{abstract}   
\section{Abstract}
	Lidar datasets now commonly reach Billions of points, as their density is often very large. 
	Using these point cloud becomes challenging, as the high number of points is untractabel for most applications and for visualisation.
	In this work we propose a new paradigm to easily get a portable geometric Level Of Details (LOD) inside a Point Cloud Server.
	The main idea is to not store the LOD information in an external additional file, but instead to store it implicitly by exploiting the order of the points.
	The point cloud is divided into groups (patches). These patches are ordered so that theim in order gradually provides more and more details on the patch. 
	We demonstrate the interest of our method with several classical uses of LOD, such as visualisation of massive point cloud, algorithm acceleration,  fast density peak detection and correction.
	Furthermore, our implicit LOD method also embeds information about the sensed object geometric nature, forming a crude multi-scale dimensionality descriptor.
	We demonstrate the interest of this descriptor by successfully using it for fast classification(basic classes) and on-the-fly filtering. 
	 
%\end{abstract}	
	